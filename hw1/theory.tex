\documentclass[]{book}


\usepackage[utf8]{inputenc}

\usepackage[english, russian]{babel}

%These tell TeX which packages to use.
\usepackage{array,epsfig}
\usepackage{amsmath}
\usepackage{amsfonts}
\usepackage{amssymb}
\usepackage{amsxtra}
\usepackage{amsthm}
\usepackage{mathrsfs}
\usepackage{color}

%Here I define some theorem styles and shortcut commands for symbols I use often
\theoremstyle{definition}
\newtheorem{defn}{Definition}
\newtheorem{thm}{Theorem}
\newtheorem{cor}{Corollary}
\newtheorem*{rmk}{Remark}
\newtheorem{lem}{Lemma}
\newtheorem*{joke}{Joke}
\newtheorem{ex}{Example}
\newtheorem*{soln}{Solution}
\newtheorem{prop}{Proposition}

\newcommand{\lra}{\longrightarrow}
\newcommand{\ra}{\rightarrow}
\newcommand{\surj}{\twoheadrightarrow}
\newcommand{\graph}{\mathrm{graph}}
\newcommand{\bb}[1]{\mathbb{#1}}
\newcommand{\Z}{\bb{Z}}
\newcommand{\Q}{\bb{Q}}
\newcommand{\R}{\bb{R}}
\newcommand{\C}{\bb{C}}
\newcommand{\N}{\bb{N}}
\newcommand{\M}{\mathbf{M}}
\newcommand{\m}{\mathbf{m}}
\newcommand{\MM}{\mathscr{M}}
\newcommand{\HH}{\mathscr{H}}
\newcommand{\Om}{\Omega}
\newcommand{\Ho}{\in\HH(\Om)}
\newcommand{\bd}{\partial}
\newcommand{\del}{\partial}
\newcommand{\bardel}{\overline\partial}
\newcommand{\textdf}[1]{\textbf{\textsf{#1}}\index{#1}}
\newcommand{\img}{\mathrm{img}}
\newcommand{\ip}[2]{\left\langle{#1},{#2}\right\rangle}
\newcommand{\inter}[1]{\mathrm{int}{#1}}
\newcommand{\exter}[1]{\mathrm{ext}{#1}}
\newcommand{\cl}[1]{\mathrm{cl}{#1}}
\newcommand{\ds}{\displaystyle}
\newcommand{\vol}{\mathrm{vol}}
\newcommand{\cnt}{\mathrm{ct}}
\newcommand{\osc}{\mathrm{osc}}
\newcommand{\LL}{\mathbf{L}}
\newcommand{\UU}{\mathbf{U}}
\newcommand{\support}{\mathrm{support}}
\newcommand{\AND}{\;\wedge\;}
\newcommand{\OR}{\;\vee\;}
\newcommand{\Oset}{\varnothing}
\newcommand{\st}{\ni}
\newcommand{\wh}{\widehat}

%Pagination stuff.
\setlength{\topmargin}{-.3 in}
\setlength{\oddsidemargin}{0in}
\setlength{\evensidemargin}{0in}
\setlength{\textheight}{9.in}
\setlength{\textwidth}{6.5in}
\pagestyle{empty}


\DeclareMathOperator*{\argmax}{argmax}
\DeclareMathOperator*{\argmin}{argmin}

\begin{document}


\begin{center}
{\Large Машинное обучение -- 1. Теоретические задачи}\\
\textbf{Каргальцев Степан}\\ %You should put your name here
26.02.2017 %You should write the date here.
\end{center}

\begin{enumerate}
\item\label{4.1} \textbf{Наивный байес и центроидный классификатор}

\textit{Покажите, что если в наивном байесовском классификаторе классы имеют одинаковые апри-
орные вероятности, а плотность распределения признаков в каждом классе имеет вид 
$P(x^{(k)} | y) = \frac{1}{\sqrt{2\pi\sigma^2}}e^{-\frac{(x^{(k)} - \mu_{yk})^2}{2\sigma^2}}, x^(k), k = 1 \ldots n$ --- признаки объекта $x$, классификация сводится к отнесению объекта $x$ к классу $y$, центр которого $\mu_y = (\mu_1, \ldots, \mu_n)^T$ ближе всего к $x$.}

$$\hat{y} = \argmax\limits_y \left [ \displaystyle\prod\limits_{k = 1}^n P(x^{(k)} | y) P(y) \right ] = |P(y) \; \textup{одинаковы}| = \argmax_y \left [\displaystyle\prod_{k = 1}^n P(x^{(k)} | y) \right ] = $$

$$\argmax\limits_y \left [\displaystyle\prod_{k = 1}^n \ln P(x^{(k)} | y)\right ] = \argmax_y \left [-\sum\limits_{k = 1}^n \frac{(x^{(k)} - \mu_{yk})^2}{2\sigma^2} - \frac{1}{2}\ln(2\pi \sigma^2) \right ] = \argmin\limits_y \left [ \sum\limits_{k = 1}^{n} (x^{(k)} - \mu_{yk})^2 \right ] = $$ 

$$\argmin\limits_y \rho(x, \mu_y) $$


\item\label{4.2} \textbf{ROC-AUC случайных ответов}

Покажите, что <<треугольный ROC-AUC>> (см.лекцию 2) в случае, когда классификатор дает
случайные ответы -- $a(x) = 1$ с вероятностью $p$ и $a(x) = 0$ с вероятностью $1 - p$, будет в
среднем равен $0.5$, независимо от $p$ и доли класса $1$ в обучающей выборке.

Если пороговое значение 0 (то есть все ответы положительны), то мы получаем TPR = 1, FPR = 1 и точку (1, 1)

Если пороговое значение 1 (то есть все ответы отрицательны), то мы получаем TPR = 0, FPR = 0 и точку (0, 0)

Иначе мы получаем TPR = $\frac{\sum\limits_{x \in \mathbbm{1}} a(x)}{|\mathbbm{1}|} =: \mathcal{Y}$, FRP = $\frac{\sum\limits_{x \in \mathbbm{0}} a(x)}{|\mathbbm{0}|} =: \mathcal{X}$.

Площадь треугольника на этих трех точках равна $S = \frac{1}{2}(\mathcal{X}, \mathcal{Y}) \times (1, 1) = \frac{1}{2} (\mathcal{X} - \mathcal{Y}) = \frac{\sum\limits_{x \in \mathbbm{0}} a(x)}{2|\mathbbm{0}|} - \frac{\sum\limits_{x \in \mathbbm{1}} a(x)}{2|\mathbbm{1}|}$.

$ES = \frac{1}{2} E \left [ \frac{\sum\limits_{x \in \mathbbm{0}} a(x)}{|\mathbbm{0}|} - \frac{\sum\limits_{x \in \mathbbm{1}} a(x)}{|\mathbbm{1}|} \right ] =  \frac{\sum\limits_{x \in \mathbbm{0}} E a(x)}{2|\mathbbm{0}|} - \frac{\sum\limits_{x \in \mathbbm{1}} E a(x)}{2|\mathbbm{1}|} = \frac{1}{2} (E a(x) - E a(x)) = 0$

То есть площадь под $ROC-AUC$ кривой будет в среднем равна $E(0.5 + S) = 0.5$

\item\label{4.3} \textbf{Ошибка 1NN и оптимального байесовского классификатора}

\textit{Утверждается, что метод одного ближайшего соседа асимптотически (при условии, что максимальное по всем точкам выборки расстояние до ближайшего соседа стремится к нулю) имеет
матожидание ошибки не более чем вдвое больше по сравнению с оптимальным байесовским
классификатором (который это матожидание минимизирует).}

\textit{Покажите это, рассмотрев задачу бинарной классификации. Достаточно рассмотреть вероятность ошибки на фиксированном объекте $x$, т.к. матожидание ошибок на выборке размера $V$ будет просто произведением $V$ на эту вероятность. Байесовский классификатор ошибается на объекте $x$ с вероятностью:}

$$E_B = \min\{P(1|x), P(0|x)\}$$

\textit{Условные вероятности будем считать непрерывными функциями от $x \in \mathbb{R}^m$   , чтобы иметь возможность делать предельные переходы. Метод ближайшего соседа ошибается с вероятно-
стью:}

$$E_N = P(y \ne y_n)$$

\textit{Здесь $y$  --- настоящий класс $x$, а $y_n$ --- класс ближайшего соседа $x_n$ к объекту $x$ в предположении, что в обучающей выборке $n$ объектов, равномерно заполняющих пространство.}

\textit{Докажите исходное утверждение, выписав выражение для $E_N$ (принадлежность к классам
$0$ и $1$ для объектов $x$ и $x_n$ считать независимыми событиями) и осуществив предельный
переход по $n$.}

$$E_N = P(1|x_n) \cdot P(0|x) + P(0|x_n) \cdot P(1|x) \to 2P(0|x)P(1|x) = $$

$$2\min \{P(0|x), P(1|x)\} \cdot \max \{P(0|x), P(1|x)\} \leqslant 2\min \{P(0|x), P(1|x)\} = 2E_B$$

\end{enumerate}
\end{document}



